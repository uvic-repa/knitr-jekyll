
\begin{abstract}
%TCIMACRO{\TeXButton{Begin single space}{\begin{singlespace}}}%
%BeginExpansion
%\begin{singlespace}%
%EndExpansion
In this paper we develop an original approach to evaluate the costs and benefits associated to a generic promotion program using an application to Bordeaux wines. The benefit is computed from the marginal impact of the collective reputation of the program on the individual reputation of its members. These different marginal impacts are estimated using detailed survey data about the image of Bordeaux wines in seven European countries. We find positive and significant spillover effects from the umbrella reputation (Bordeaux) that moreover increase with the individual reputation level of the wine. Controlling for the natural endogeneity of the collective reputation in this setup, we capture the important fact that this relationship is faced with marginal diminishing returns. These spillover effects, when significantly positive, vary from a minimum of 5\% to a maximum of 15\% of additional favorable quality opinions. We then show that some subregions are more likely to benefit from generic promotion programs, suggesting that fees should be established on a benefit-cost basis.

\textbf{Key Words:} Benefit-cost analysis, Individual reputation, Collective reputation, Bordeaux wines, Appellations.

\textbf{JEL Classification: }L15 - L66 - Q13 - Z13 
%TCIMACRO{\TeXButton{End Single Space}{\end{singlespace}}}%
%BeginExpansion
%\end{singlespace}%
%EndExpansion
\end{abstract}