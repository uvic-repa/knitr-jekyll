\section{Discussion}
\label{sec:Discussion}

Given the importance of electricity to industrial economies, and because of increasing emphasis on using battery-powered, hybrid and/or fuel-cell vehicles that require electricity, there has been a great deal of interest in promoting wind energy. In this study, we examined the potential to replace coal-fired power in Alberta with wind energy. Since wind regimes play a very important role, we used wind speed data from locations scattered widely across the province (in some cases a thousand or more kilometers apart) to develop wind power regimes for the decade 2006-2015. Then, using 2015 load and infrastructure data for Alberta, we examined the potential for wind energy to reduce greenhouse gas emissions. Our findings indicate that variability of wind speeds from one hour to the next and from one year to the next has a great impact on the viability of investments in wind energy. For some wind regimes, investments in wind power make sense without further incentives; indeed, we found this to be the case for the winds that characterize southwestern Alberta around Pincher Creek where most of Alberta’s existing wind farms are located. For other wind regimes, incentives are needed to induce investment in wind power. With the exception of certain locations such as Pincher Creek, the variability in wind regimes militates against investment in wind turbines.


We also considered solar power in Alberta but found that, based on the available data, solar power was sufficiently inadequate during winter and night times to warrant consideration at this time. If better data on solar radiation and photovoltaic conversion become available, this will need to be considered further. However, since solar only accounts for some 2\% of total renewable capacity and has been shown to have a capacity factor of only 11\% in Germany, it is unlikely that solar PV can overcome the problems identified here, particularly the high costs of implementing wind power in areas outside of a small region in southwestern Alberta. 



Finally, if politicians in Alberta are serious about reducing greenhouse gas emissions by 30\% or more and, at the same time, continue to develop the oil sands despite a cap on annual emissions of 100 Mt CO2, it is unlikely this can be achieved without purchasing carbon offsets outside the province or investing in nuclear power. Given that prices of carbon offsets are likely to rise exorbitantly in the future as more and more jurisdictions look to carbon offsets to meet emission reduction targets, and as developing countries are brought into an effective emission-reduction agreement, the most realistic option might well be a nuclear one. Planning should at least consider this option. 
