\section{Alberta Electricity Grid}
\label{sec:AlbertaElectricityGrid}

The Alberta electricity grid is characterized by industrial consumers and three main types of generation – coal, natural gas and co-generation. The 2015 load duration curve shown in Figure 1 is indicative of the province’s industrial base; the peak load of 11,229 MW is only 56\% greater than the baseload of 7,203 MW, and baseload demand (63.10 TWh) accounts for 78.6\% of total generation of 80.26 TWh. In contrast, for example, the peak loads of British Columbia and Ontario are more than double those of baseload. Alberta’s generation mix is dominated by fossil fuels (Table 2) despite recent efforts to increase use of biomass and wind (there is no solar in the mix), and recovery of waste heat. As indicated in Figure 2, installed wind capacity has increased by 1,445 MW since 2000, while co-generation capacity increased by 2,951 MW (most of which relies on natural gas and not biomass), natural gas plant capacity by 1,372 MW, and coal-fired capacity by 556 MW, although overall investment in coal capacity has been greater since over 500 MW was decommissioned during the same period
 
 
 
Figure 1: Alberta Load Duration Curve, 2015



% Please add the following required packages to your document preamble:
% \usepackage{booktabs}
\begin{table}[tbp]
\centering

\begin{threeparttable}
\caption{\bf Capacity and Generation, Alberta Electric System, 2014}
\label{my-label}
\begin{tabular}{@{}llllll@{}}
\toprule
            & \multicolumn{2}{l}{Capacity} &  & \multicolumn{2}{l}{Generation} \\ \midrule
Fuel Source & MW           & Share         &  & GWh           & Share          \\
Coal        & 6,258        & 38.5\%        &  & 44,442        & 55.0\%         \\
Natural Gas & 7,080        & 43.6\%        &  & 28,136        & 35.0\%         \\
Hydro       & 900          & 5.5\%         &  & 1,861         & 2.0\%          \\
Wind        & 1,459        & 9.0\%         &  & 3,471         & 4.0\%          \\
Biomass\tnote{a}    & 447          & 2.8\%         &  & 2,060         & 3.0\%          \\
Other\tnote{b}      & 98           & 0.6\%         &  & 373           & 0.0\%          \\
Total       & 16,242       & 100.0\%       &  & 80,343        & 100.0\%        \\ \bottomrule
\end{tabular}

    \begin{tablenotes}
      \small
      \item[a] Co-gen biomass accounts for 158.0 MW of capacity, biogas for 8.8 MW and other biomass for the remainder.
      \item[b]  Includes fuel oil and waste heat, which is a by-product of existing industrial operations with the heat otherwise escaping from an exhaust pipe.
      \item Source: Alberta Utilities Commission (AUC) and Alberta Electric System Operator (AESO)
    \end{tablenotes}
    
\end{threeparttable}
\end{table}




 
Figure 2: Installed Generating Capacity by Type, Alberta, 2000-2015



\section{Costs of Producing Electricity}

A recent evaluation of the U.S. costs of generating electricity by Stacy and Taylor (2015) examined the costs of generating electricity by three types of assets: baseload assets capable of dispatching electricity at any time and for very long periods (coal, combined-cycle natural gas, nuclear and hydro), dispatchable peak resources (gas turbines), and intermittent resources (wind). They compare EIA (2010) estimates of LCOE based on information from existing plants, estimates of what it would cost to produce electricity from new plants with the latest technology, and estimates for new construction but revised to take into account observed capacity factors (CFs) rather than assumed CFs\footnote{\label{CFs} A generating asset’s capacity factor is given by the ratio of the annual electricity generated by the asset divided by the asset’s capacity multiplied by 8760 hours (8784 hours in a leap year). }.  Their calculations are provided in Table 2, along with more recent estimates from EIA (2015).



% Please add the following required packages to your document preamble:
% \usepackage{booktabs}
\begin{table}[]

\caption{ \bf Estimates of the Levelized Costs of Electricity (LCOE) for Existing Plants, New Construction with Optimistic Capacity Factors and New Construction based on Observed Capacity Factors, and Latest Estimates, Three Generating Asset Types (\$US 2012/MWh)}



\centering
\begin{threeparttable}

\label{my-label2}
\begin{tabular}{@{}lllll@{}}
\toprule
Generator Type                           & Existing\tnote{a}      & Optimistic\tnote{a} & Observed \tnote{a} & Latest\tnote{b} \\ \midrule
\multicolumn{5}{l}{\textbf{Dispatchable full-time capable resources (baseload)}}             \\
Conventional coal                        & 38.4          & 80.0        & 97.7      & 93.7    \\
Conventional combined cycle gas (CC gas) & 48.9          & 66.3        & 73.4      & 74.1    \\
Nuclear                                  & 29.6          & 96.1        & 92.7      & 93.8    \\
Hydro (seasonal)                         & 34.2          & 84.5        & 116.8     & 82.2    \\ \midrule
\multicolumn{5}{l}{\textbf{Dispatchable peaking resources}}                                  \\
Conventional combustion turbine (CT gas) & 142.8         & 128.4       & 362.1     & 139.4   \\ \midrule
\multicolumn{5}{l}{\textbf{Non-dispatchable intermittent resource as used in practice}}      \\
Wind                                     & Not available & 96.2        & 112.8     & 72.5    \\ \bottomrule
\end{tabular}
    \begin{tablenotes}
      \small
      \item[a]  Source: Stacy and Taylor (2015). The ‘Existing’ column is based on their own calculations. Data in the ‘Optimistic’ and ‘Observed’ columns are based on EIA (2010).
      \item[b]  Source: EIA (2015). Values for plants entering service in 2020; \$2013 values deflated to \$2012 using inflation rate of 1.5\%. Capacity factors for wind (36\%) and solar (25\%) are the best observed in the U.S., so LCOEs for intermittent resources are likely higher than reported here.
    \end{tablenotes}

\end{threeparttable}
\end{table}

The results in Table 2 indicate that current costs of producing electricity (Existing column) are much lower than those of new construction. This is primarily because the construction costs of many assets have been paid off. Decision makers need to consider this when they implement policies that result in the premature closure of existing generators, because doing so might lead to higher than expected overall electricity costs. Next, estimates of the LCOEs for new construction indicate that, despite recent advances in technology, wind remains at a cost disadvantage relative to fossil fuels, and more so if costs of additional transmission are taken into account. Finally, if the observed as opposed to estimated CF is used to calculate levelized costs, the LCOE for new construction will turn out to be higher than expected by the EIA (2010, 2015), thereby reinforcing preference for keeping current assets longer. 


There are two caveats to consider. First, the use of LCOE to select renewable energy projects (or otherwise make investment choices) can be misleading because the value of power changes over time and space, as does the production of power from various assets, especially intermittent ones. Second, LCOE estimates exclude externality costs, except perhaps in the case of nuclear power, where recent cost overruns to address evolving environmental regulations have resulted in construction delays and higher costs (Lovering et al. 2016). This issue is best addressed by employing an annualized cost (or penalty) for investing in new generating capacity, which could be over and above the carbon tax used to incentivize both investment and generation. Finally, a (quite) small penalty is imposed to incentivize removal of assets that fail to produce power during the year.


Overnight construction costs are difficult to determine. In the current analysis, we use data from surveys conducted at various times by the International Energy Agency (IEA) and U.S. Energy Information Administration (EIA)\footnote{See, e.g., http://www.eia.gov/oiaf/beck\_plantcosts/index.html [accessed May 25, 2016]. }. It should be noted, however, that our results are robust regarding capital costs.  We assume overnight costs of wind are \$2,700/kW, while those of coal, conventional combustion turbine (CT) gas (which we assume to be the same as for co-gen), combined-cycle (CC) gas, and nuclear power plants are \$2,600/kW, \$1,900/kW, \$1,600 and \$6,000/kW, respectively. Capital costs are annualized using a 5\% discount rate and the estimated length of time taken to build the facility.


Wind power is less expensive on a LCOE basis than CT gas and seasonal hydro, but more expensive than electricity generated from CC gas, nuclear and coal. Overall, available data indicate that traditional fossil fuel technologies are clearly preferred to wind power on a cost basis, unless externality costs are taken into account.


\citet{Bordo2005} from here.