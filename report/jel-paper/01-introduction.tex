\section{Introduction}
\label{sec:introduction}




Alberta’s NDP government has promised to make the province a world leader in renewable energy. To achieve this, all coal-fired electricity generation facilities are to be phased out by 2030, with two-thirds of the lost electricity production to be replaced by renewables, primarily wind and solar power, with natural gas to be used for generating baseload power (and as backup to intermittent energy sources). To encourage a speedy transition, the government will implement an economy-wide carbon tax of \$20 per tonne CO2 (tCO2) beginning in 2017 and increase it to \$30/tCO2 in 2018; commit to produce 30\% of electricity from renewables by 2030; provide subsidies to encourage renewable energy; and cap emissions from oil sands developments at 100 megatons of CO2 (Government of Alberta 2016). 


The costs of developing and operating renewable (wind- or solar-powered) generating assets can be substantial and may only be viable for firms when governments provide subsidies or other inducements. An indirect but important cost of these renewables is associated with the high variability of wind patterns and inconsistent availability of solar energy (due to lack of sunlight). Intermittency in wind (and solar) power output is unavoidable and the gaps result in large costs of ramping existing generating assets or investing in new assets to compensate for this intermittency (van Kooten 2016a).  Nonetheless, there are also significant benefits to society of transitioning away from fossil fuels, primarily from reduced greenhouse gas (GHG) emissions as measured in terms of CO2. The benefits to society mainly rely on the extent to which clean energy replaces fossil fuels, especially coal-fired power.  


To measure the degree to which intermittent energy can substitute for coal-fired power will be determined in this study by first collecting hourly wind data from various locations across the Alberta. We then assume establishment of sufficient wind generating capacity to meet half of the province’s peak load, with wind farms spread across the province to reduce the potential intermittency in supply as wind speeds vary across the landscape. We simulate the wind power that could have been generated every hour for the period 2006 through 2015 using wind-turbine power curves and the data on wind speeds. Hourly wind power output is then subtracted from demand to obtain the load that must be met by the various fossil-fuel and other generating assets comprising the Alberta electricity system. 


Our objective is to examine the case where the Alberta government seeks to eliminate coal, using both a carbon tax and regulation. To determine how generation is allocated across assets in each hour, we use an existing grid model for Alberta that optimizes load across assets (van Kooten et al. 2013)\footnote{We focus only on wind power because potential solar photovoltaic (PV) electricity output is much more difficult to model and beyond the scope of the current study. However, the inherent intermittency we model using wind is likely little affected by adding solar power. For example, Monahan and van Kooten (2010) found that adding a predictable tidal power output profile to a wind profile had no impact on the management of a power grid on Haida Gwaii off the Northwest coast of British Columbia.}. The grid allocation model is annual with an hourly time step, and assumes rational expectations on the part of the grid operator/asset owner. Any excess power remaining when intermittent wind power is subtracted from load is assumed to be exported to British Columbia, which has the ability to store power behind hydroelectric dams, or to Saskatchewan or the United States via transmission interties. Excess power is assumed to be sold at a price of zero – Alberta receives no remuneration nor does it have to pay (negative price) to dump the excess power. Costs are determined in the analysis using the levelized cost of electricity (LCOE) when power is generated from various facilities as determined by the grid allocation model, although an investor is assumed to incur an annualized cost of construction. The optimal allocation of output across assets is also used to determine CO2 emissions. 


Our analysis seeks to determine if the benefits to society from changing to wind and solar energy will outweigh the costs, and whether imposing strong restrictions on fossil fuels is economically feasible. We can calculate the costs and benefits (using different values of the social cost of carbon or shadow price of carbon) of reducing CO2 emissions from Alberta’s electricity sector. Depending on the results, it may be necessary to examine the optimal subsidies required by governments to facilitate the construction of renewable generating facilities as well as the compensation for firms that had earlier been encouraged to invest in new coal-fired generating capacity. 
We begin in the next section by examining the characteristics of the Alberta electricity system, followed by a discussion of the costs of producing electricity. Then we describe our model and the origins of the wind power data that we employ. This is followed by our results and concluding discussion.


